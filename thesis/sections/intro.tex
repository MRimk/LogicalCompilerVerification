\chapter{Introduction}\label{s:intro}



% Compilers are a crucial part of each computer and system since each static programming language relies on the correct abstraction of machine code and execution. This is where formal verification comes into effect. 

% Lean is a powerful tool to keep the proof correct since it verifies itself rather than a pen-and-paper proof. This means that mistakes are easy to spot and big-proof systems could be built on top of each other ground-up, which would not be as straightforward in a formal text-written proof. There are other theorem provers such as Isabelle/HOL, \todo{insert other provers}, but Lean has the advantage of \todo{insert why I am doing this in lean}

% In this paper, I will be creating a simple stack machine language - WHILE \cite{hithchiker}- compiler, and proving its correctness in LEAN. LEAN is a functional programming language that is used as a theorem prover to derive logical verification for its functions and types. \todo{expand}

% This thesis is heavily inspired by Chapter 8 of Concrete Semantics: With Isabelle/HOL \cite{isabelle}. It builds upon a simple imperative language defined in Chapter 8 of Hitchhiker's Guide to Logical Verification \cite{hithchiker}.

% \todo{talk about why and how I am contributing}

